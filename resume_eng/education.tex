\cvsection{Education}

\begin{cventries}

	\cventry%
		{Coursera, Stanford University, MITx, LinkedIn Learning}
		{Self-taught and MOOCs}
		{}
		{2011-ongoing}
		{\begin{cvitems}
			\item {With distinction:
				Machine Learning,
				Practical Machine Learning,
				Pattern Discovery,
				Databases,
				Artificial Intelligence for Robotics,
				Algorithms
				}
			\item {%
				IT Security Specialist,
				Agile Software Development,
				Computer Vision,
				Deep Learning,
				Process Mining,
				Natural Language Processing,
				Social Network Analysis,
				Probabilistic Graphical Models,
				Automata,
				Control of Mobile Robots,
				Quantum Mechanics and Computing,
				Power Electronics
				}
		\end{cvitems}}

	\cventry%
		{Università degli Studi di Padova}
		{MSc in Automation Engineering}
		{Padua (IT)}
		{2013}
		{\begin{cvitems}
			\item {Thesis: \emph{RUUMBA: a Range-only, Unscented, Undelayed, Mobile Beacon-Assisted framework for WSN discovery and localization}.}
			\item {%
				Gaussian Mixture, Kalman, and particle filters;
				Wireless Sensor Networks;
				operative research;
				computer vision;
				robotics;
				distributed computing and consensus;
				information theory;
				identification, prediction and control of dynamic systems;
				random processes.
				}
			% \item sup.~prof.~Andrea Zanella, PhD
			% \item {93/110}
			% \item {A Wireless Sensor Network was localized using only Received Signal Strength measurements between nodes: this was accomplished by dynamical path planning of an autonomous robot mobile beacon, to guarantee the redundant rigidity of the underlying distance graph. Unscented Kalman Filters and aggressive pruning of Gaussian Mixture Models were used to improve computational and energy efficiency.}
		\end{cvitems}}

	\cventry%
		{Università degli Studi di Padova}
		{B.S. in Automation Engineering}
		{Padua (IT)}
		{2008}
		{\begin{cvitems}
			\item {Thesis: \emph{Development of a SCADA software for management and control of a co-generation power plant}.}
			\item {%
				Industrial Control Systems, SCADA and PLC programming;
				linear and nonlinear signal processing;
				time and frequency domain analysis;
				PID process control;
				object-oriented and functional programming;
				probability and statistics;
				linear algebra;
				electronics and telecommunications.
				}
			% \item sup.~prof.~Stefano Vitturi
			% \item {90/110}
			% \item {This thesis was connected to an internship at Teletronic S.r.l., during which both a Siemens PLC and its connected Human-Machine Interface were developed to automatize, supervise and control a biomass power plant for electricity and heat co-generation. A novel "key-value" solution was designed and implemented to efficiently update alarm thresholds.}
		\end{cvitems}}

\end{cventries}
