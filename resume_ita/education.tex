\cvsection{Educazione}

\begin{cventries}

	\cventry%
		{Università degli Studi di Padova}
		{Laurea Magistrale in Ing. dell'Automazione}
		{Padova (IT)}
		{2013}
		{\begin{cvitems}
			\item {Tesi: \emph{RUUMBA: a Range-only, Unscented, Undelayed, Mobile Beacon-Assisted framework for WSN discovery and localization}.}
			\item {%
				Filtri Gaussian Mixture, di Kalman, e particellari;
				Wireless Sensor Networks;
				ricerca operativa;
				visione computazionale;
				robotica;
				calcolo distrubuito e consensus;
				teoria dell'informazione;
				identificazione, stima e controllo di sistemi dinamici;
				processi stocastici.
			}
			% \item sup.~prof.~Andrea Zanella, PhD
			% \item {93/110}
			% \item {A Wireless Sensor Network was localized using only Received Signal Strength measurements between nodes: this was accomplished by dynamical path planning of an autonomous robot mobile beacon, to guarantee the redundant rigidity of the underlying distance graph. Unscented Kalman Filters and aggressive pruning of Gaussian Mixture Models were used to improve computational and energy efficiency.}
		\end{cvitems}}

	\cventry%
		{Università degli Studi di Padova}
		{Laurea in Ing. dell'Automazione}
		{Padova (IT)}
		{2008}
		{\begin{cvitems}
			\item {Tesi: \emph{Sviluppo di un software SCADA per controllo e gestione di un impianto di cogenerazione}.}
			\item {%
				Automazione industriale, PLC e SCADA;
				analisi di segnali lineari e nonlineari nel tempo e in frequenza;
				controllo dei processi tramite PID;
				programmazione ad oggetti e funzionale;
				probabilità e statistica;
				algebra lineare;
				elettronica e telecomunicazioni.
				}
			% \item sup.~prof.~Stefano Vitturi
			% \item {90/110}
			% \item {This thesis was connected to an internship at Teletronic S.r.l., during which both a Siemens PLC and its connected Human-Machine Interface were developed to automatize, supervise and control a biomass power plant for electricity and heat co-generation. A novel "key-value" solution was designed and implemented to efficiently update alarm thresholds.}
		\end{cvitems}}

	\cventry%
		{Coursera, Stanford University, MITx, LinkedIn Learning}
		{Autodidatta e MOOCs}
		{}
		{2011-ongoing}
		{\begin{cvitems}
			\item {Con distinzione:
				Machine Learning,
				Practical Machine Learning,
				Pattern Discovery,
				Databases,
				Artificial Intelligence for Robotics,
				Algorithms
				}
			\item {%
				IT Security Specialist,
				Agile Software Development,
				Computer Vision,
				Deep Learning,
				Process Mining,
				Natural Language Processing,
				Social Network Analysis,
				Probabilistic Graphical Models,
				Automata,
				Control of Mobile Robots,
				Quantum Mechanics and Computing,
				Power Electronics
				}
		\end{cvitems}}
\end{cventries}
